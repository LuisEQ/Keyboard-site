\documentclass{scrreprt}
\usepackage[spanish]{babel}
\usepackage{listings}
\usepackage{underscore}
\usepackage[bookmarks=true]{hyperref}
\usepackage[utf8]{inputenc}
\usepackage{hyperref}
\usepackage{float,graphics,epsfig}

\hypersetup{
	bookmarks=false,
	pdftitle={Especificación de Requerimientos de Software},
	pdfauthor={Gerardo Tinoco Guerrero},
	colorlinks=true,
	linkcolor=blue,
	citecolor=black,
	filecolor=black,
	urlcolor=purple,
	linktoc=page
}

\def\myversion{1.0 }

\begin{document}

	\begin{flushright}
		\rule{16cm}{5pt}\vskip1cm
		\begin{bfseries}
			\huge{ESPECIFICACIÓN DE REQUERIMIENTOS DE SOFTWARE}\\
			\vspace{1.9cm}
			para\\
			\vspace{1.9cm}
			``Proyecto final de página web de tutoriales de teclados"\\
			\vspace{1.9cm}
			\LARGE{Version \myversion}\\
			\vspace{1.9cm}
			Preparado por Gerardo Tinoco Guerrero\\
			\vspace{1.9cm}
			Realizado por Luis Eduardo Quintero López\\
			\vspace{1.9cm}
			\today\\
		\end{bfseries}
	\end{flushright}
	
	\tableofcontents
	

	
	\chapter*{Historial de Revisiones}

		\begin{center}
			\begin{tabular}{|c|c|c|c|}
				\hline
				Nombre	&	Fecha	&	Razón de los cambios	&	Versión\\
				\hline
				Luis Eduardo Quintero López	& 	\today 	& 	Primera iteración 				& 	1\\
				\hline
			\end{tabular}
		\end{center}
		
	\chapter{Introducción}
		Este proyecto es una página web realizada con React y Gatsby. Gatsby es un nuevo enfoque para crear aplicaciones web, basado en tecnologías modernas como React y graphql, que te permiten construir sitios web que cargan y se ejecutan de manera muy rápida. Esto nos sirve para el enfoque que tendremos en la página web, que es una wiki y tutoriales del tema de teclados mecánicos custom, esta en español únicamente e intentando utilizar un español neutro para que pueda ser leída por una persona de cualquier país de habla hispana.\\
¿Por qué una página web? En este formato nos podría ser más fácil que cualquier persona pueda leer el contenido sin tener que descargar nada y la complejidad de esta no es lo suficiente como para justificar una aplicación, también al usar Gatsby nos da la ventaja de que todo carga rápido para el usuario y se siente responsiva desde el momento que se abre. \\
\\
La razón por la que quiero hacer una página web de teclados, es para ayudar a que el hobby crezca, que tengamos dentro del nicho información verídica en español, ya que como no lleva tantos años, hay muy poca información y esta solo se encuentra en inglés, coreano y japonés y es escasa, por lo que de esta forma se tendría la primer página en español de ello, alentaría a la gente que no sabe uno de estos idiomas pero le interesa el tema en meterse al hobby sin la necesidad de tener que estar todo el tiempo utilizando traductores u otras cosas.\\

		\begin{figure}[htbp]
			\begin{center}
				\includegraphics[width = 0.5\textwidth , height = 5cm]{3}
				\caption{Imagen de nazaré 1-60 de Nazaré Engineering}
				\label{referencia}
			\end{center}
		\end{figure}

		
		%{Por ejemplo, sigo escribiendo aquí. En la Figura \ref{referencia} podemos ver una imagen.
		
		\section{Propósito}
		Este proyecto tiene como propósito iniciar mi proyecto personal de hacer esta página web ya en un framework, ya que anteriormente estaba realizada con notion, que funciona pero no es del todo agradable a la vista y es algo lenta, también la personalización es casi nula pero la ventaja de tenerlo en notion es que hacer el setup y tener una base es muy sencilla gracias a las plantillas y el cómo funciona notion. Utilizando React y Gatsby se tiene una página muy rápida, que esto es una gran ventaja contra notion, también al hacer la página desde 0 se tiene el poder de modificar el diseño por completo, algo que ayudará a la estética bastante a comparación de con notion.\\
		%\section* {Esta sección no aparece porque tiene un asterisco( no aparece en el indicie y no esta ennumerado)} %LEER ESTO
	
		\section{Público objetivo y sugerencias de lectura}
	El público objectivo es aquellos que necesitan trabajar en el proyecto ya sea dentro del frontend, backend o que necesitan trabajar con el proyecto de una manera más externa.
Se sugiere se lea toda la introducción para poder entender cómo va y se entiendan las bases del proyecto para poder entender las siguientes secciones que ya dan por entendido que se sabe qué es lo que se está viendo. Después de la introducción es posible irse a la sección en la que se necesita la información, de esta manera ya se entenderá fácil el tema tratado.\\
	
		\section{Alcance del proyecto}
	El alcance que contiene es tener cumplidos los requisitos funcionales y no funcionales marcados al inicio del proyecto.
	
	
	\chapter{Descripción general}
	
		\section{Perspectiva del producto}
	Esta página será utilizada para poder ayudar a gente que quiera meterse en el hobby de teclados mecánicos y poder entender las bases de esto en una sola página y no tener que pasar por información que probable es de boca a boca y no es confiable del todo. Esta página estaría patrocinada por un vendedor de piezas de teclados de México y por la página de servicios de importación (proxy) llamada Houndbox. Usualmente cuando alguien nuevo al hobby se quiere meter, es tener que preguntar a amigos o grupos sobre el tema, con el uso de la página se quitaría esa gran barrera que hay.
	
		\section{Funciones del producto}
Requisitos funcionales:
\begin{itemize}
	\item Buscador.
	\item La página usará un navbar para poder acceder fácil a diferentes secciones de la pagina web.
	\item La página permite a los usuarios moverse entre categorías diferentes.
	\item La página web permite ver todas las secciones disponibles dentro del navbar.
	\item El sitio permite que los administradores agreguen nuevo contenido o editen contenido anterior.
	\item El front page tendrá información importante, junto con links a artículos populares.
	\item Tendrá un botón para cambiar la página a dark mode o light.
	\item Pagina responsiva
\end{itemize}
Requisitos no funcionales:
\begin{itemize}
	\item Velocidad de carga rápida
	\item Documentación en el proyecto
	\item Reusabilidad
	\item Estabilidad
	\item Los colores a usar será para los acentos azul, mientras que la página será utilizando blanco de fondo o gris.
\end{itemize}
	
		\section{Clases de usuario y características}
		Los siguientes diagramas son los diagramas del proyecto que fueron utilizados para su desarrollo:\\
		\begin{figure}[htbp]
			\begin{center}
				\includegraphics{1}
				\caption{Imagen de nazaré 1-60 de Nazaré Engineering}
				\label{referencia}
			\end{center}
		\end{figure}

		\begin{figure}[htbp]
			\begin{center}
				\includegraphics{2}
				\caption{Diagrama}
				\label{referencia}
			\end{center}
		\end{figure}

		\begin{figure}[htbp]
			\begin{center}
				\includegraphics{33}
				\caption{Diagrama}
				\label{referencia}
			\end{center}
		\end{figure}

		\begin{figure}[htbp]
			\begin{center}
				\includegraphics{4}
				\caption{Diagrama}
				\label{referencia}
			\end{center}
		\end{figure}

		\begin{figure}[htbp]
			\begin{center}
				\includegraphics{5}
				\caption{Diagrama}
				\label{referencia}
			\end{center}
		\end{figure}

		\begin{figure}[htbp]
			\begin{center}
				\includegraphics{6}
				\caption{Diagrama}
				\label{referencia}
			\end{center}
		\end{figure}

		\begin{figure}[htbp]
			\begin{center}
				\includegraphics{7}
				\caption{Diagrama}
				\label{referencia}
			\end{center}
		\end{figure}

		\begin{figure}[htbp]
			\begin{center}
				\includegraphics{8}
				\caption{Diagrama}
				\label{referencia}
			\end{center}
		\end{figure}

		\begin{figure}[htbp]
			\begin{center}
				\includegraphics{9}
				\caption{Diagrama}
				\label{referencia}
			\end{center}
		\end{figure}

		\section{Entorno operativo}

	
		\section{Restricciones de diseño e implementación}

	
		\section{Documentación del usuario}
	
	\chapter{Requisitos de la interfaz externa}
	
		\section{Interfaces de usuario}
		
		\section{Interfaces de hardware}
		
		\section{Interfaces de software}
		
		\section{Interfaces de comunicaciones}

	\chapter{Características del sistema}
		\section{Característica del sistema 1}
			
			\subsection{Descripción y Prioridad}
El sistema está realizado con React y Gatsby, por lo que la página es manejada con javascript, typescript, html, css y también hace uso de graphql. Las prioridades en el poryecto es la realización de los blogs que tendrán la información, así como la interfaz visual de lo que es el navbar y las páginas principales; una de las secciones de menor prioridad es la sección de tutoriales, que tendrá el inicio de sesión junto con la información guardada del usuario de cómo va en sus tutoriales.
			
			\subsection{Requerimientos funcionales}
Requisitos funcionales:
\begin{itemize}
	\item Buscador.
	\item La página usará un navbar para poder acceder fácil a diferentes secciones de la pagina web.
	\item La página permite a los usuarios moverse entre categorías diferentes.
	\item La página web permite ver todas las secciones disponibles dentro del navbar.
	\item El sitio permite que los administradores agreguen nuevo contenido o editen contenido anterior.
	\item El front page tendrá información importante, junto con links a artículos populares.
	\item Tendrá un botón para cambiar la página a dark mode o light.
	\item Pagina responsiva
\end{itemize}
		
	\chapter{Otros requisitos no funcionales}
	
		\section{Requisitos de desempeño}
\begin{itemize}
	\item Se requiere que sea rápida.
	\item Que las páginas a las que se puede ingresar con un click, carguen desde que el botón o link esté en pantalla para que si el usuario quiere ingresar, sienta como si fuera de inmediato el cambio de página.
\end{itemize}
		
		
		\section{Requerimientos de seguridad}
\begin{itemize}
	\item Que cumpla con los estándares de seguridad https.
	\item Que cumpla los estándares de seguridad para el acceso a cuentas.
	\item La información del usuario será utilizada solo para fines de acceso a cuentas.
\end{itemize}
		
		\section{Atributos de calidad del software}
\begin{itemize}
	\item Documentación dentro de cada archivo de código.
	\item Dividir el código lo más que se pueda para poder ser reutilizado de llegar a ser posible.
\end{itemize}
		
		\section{Reglas del negocio}
\begin{itemize}
	\item Las actualizaciones a la página serán mensuales.
	\item No se venderá información.
	\item Dar información a los lectores de los cambios próximos o blogs que se tienen planeados.
	\item La información del usuario será utilizada solo para fines de acceso a cuentas.
\end{itemize}


	\chapter*{Apéndice A: Glosario}
El frontend es la parte del desarrollo web que se dedica a la parte frontal de un sitio web, en pocas palabras del diseño de un sitio web, desde la estructura del sitio hasta los estilos como colores, fondos, tamaños hasta llegar a las animaciones y efectos.\\

HTML (lenguaje de marcado de hipertexto, se utilizan etiquetas que estructuran y organizan el contenido de la web). \\

CSS (se encarga del formato y diseño visual de las páginas web escritas en html) para darle estructura y estilo al sitio. \\

Javascript (un lenguaje de programación, rápido y seguro para programar centros de datos, consolas, teléfonos móviles o Internet) para complementar los anteriores y darle dinamismo a los sitios web. \\

Backend es la capa de acceso a los datos, ya sea de un software o de un dispositivo en general, es la lógica tecnológica que hace que una página web funcione, lo que queda oculto a ojos del visitante.\\

Framework es un esquema o marco de trabajo que ofrece una estructura base para elaborar un proyecto con objetivos específicos, una especie de plantilla que sirve como punto de partida para la organización y desarrollo de software.\\
		
	\chapter*{Apéndice B: Modelos de análisis}

\end{document}